\documentclass[]{book}

%These tell TeX which packages to use.
\usepackage{array,epsfig}
\usepackage{amsmath}
\usepackage{amsfonts}
\usepackage{amssymb}
\usepackage{amsxtra}
\usepackage{amsthm}
\usepackage{mathrsfs}
\usepackage{color}

% JORDAN ADDED THESE
\usepackage{enumerate}
\usepackage[shortlabels]{enumitem}

\usepackage[labelsep=period,singlelinecheck=false]{caption}   %%% Used to manipulate captions, puts period after figure number, and aligns captions to the left. %%%
\usepackage{tabularx}  		%%% Allows you to construct tables that extend to page margins %%%
\usepackage{subcaption} 			%%% used to manipulate subcaptions %%%

\usepackage{algorithmic}
\usepackage{algorithm}
\usepackage{multirow}
% -----------------------------------------


%Here I define some theorem styles and shortcut commands for symbols I use often
\theoremstyle{definition}
\newtheorem{defn}{Definition}
\newtheorem{thm}{Theorem}
\newtheorem{cor}{Corollary}
\newtheorem*{rmk}{Remark}
\newtheorem{lem}{Lemma}
\newtheorem*{joke}{Joke}
\newtheorem{ex}{Example}
\newtheorem*{soln}{Solution}
\newtheorem{prop}{Proposition}

\newcommand{\lra}{\longrightarrow}
\newcommand{\ra}{\rightarrow}
\newcommand{\surj}{\twoheadrightarrow}
\newcommand{\graph}{\mathrm{graph}}
\newcommand{\bb}[1]{\mathbb{#1}}
\newcommand{\Z}{\bb{Z}}
\newcommand{\Q}{\bb{Q}}
\newcommand{\R}{\bb{R}}
\newcommand{\C}{\bb{C}}
\newcommand{\N}{\bb{N}}
\newcommand{\M}{\mathbf{M}}
\newcommand{\m}{\mathbf{m}}
\newcommand{\MM}{\mathscr{M}}
\newcommand{\HH}{\mathscr{H}}
\newcommand{\Om}{\Omega}
\newcommand{\Ho}{\in\HH(\Om)}
\newcommand{\bd}{\partial}
\newcommand{\del}{\partial}
\newcommand{\bardel}{\overline\partial}
\newcommand{\textdf}[1]{\textbf{\textsf{#1}}\index{#1}}
\newcommand{\img}{\mathrm{img}}
\newcommand{\ip}[2]{\left\langle{#1},{#2}\right\rangle}
\newcommand{\inter}[1]{\mathrm{int}{#1}}
\newcommand{\exter}[1]{\mathrm{ext}{#1}}
\newcommand{\cl}[1]{\mathrm{cl}{#1}}
\newcommand{\ds}{\displaystyle}
\newcommand{\vol}{\mathrm{vol}}
\newcommand{\cnt}{\mathrm{ct}}
\newcommand{\osc}{\mathrm{osc}}
\newcommand{\LL}{\mathbf{L}}
\newcommand{\UU}{\mathbf{U}}
\newcommand{\support}{\mathrm{support}}
\newcommand{\Oset}{\varnothing}
\newcommand{\st}{\ni}
\newcommand{\wh}{\widehat}

%Pagination stuff.
\setlength{\topmargin}{-.3 in}
\setlength{\oddsidemargin}{0in}
\setlength{\evensidemargin}{0in}
\setlength{\textheight}{9.in}
\setlength{\textwidth}{6.5in}
\pagestyle{empty}



\begin{document}


\begin{center}
{\Large COMP 775 - Advanced Design and Analysis of Algorithms \hspace{0.5cm} Assignment 3}\\
\textbf{Jordan Sturtz}\\ %You should put your name here
3-17-2023 %You should write the date here.
\end{center}

\vspace{0.2 cm}

\subsection*{Problem 4-5}

\textbf{Question} 
Professor Diogenes has n supposedly identical integrated-circuit chips that in principle are capable of testing each other. 
The professor’s test jig accommodates two
chips at a time. When the jig is loaded, each chip tests the other and reports whether
it is good or bad. A good chip always reports accurately whether the other chip is
good or bad, but the professor cannot trust the answer of a bad chip. Thus, the four
possible outcomes of a test are as follows:

\begin{table}[H]
\captionsetup{labelfont=it,justification=centering} 		%%% Puts label font in italics, must be done for all figures %%%
\begin{tabular}{l l l} % <-- Alignments: 1st column left, 2nd middle and 3rd right, with vertical lines in between
    \textbf{Chip A says} & \textbf{Chip B Says} & \textbf{Conclusion}
    \tabularnewline\hline
    B is good        & A is good & both are good, or both are bad
    \tabularnewline
    B is good        & A is bad & at least one is bad
    \tabularnewline
    B is bad        & A is good & at least one is bad
    \tabularnewline
    B is bad        & A is bad & at least one is bad
    \tabularnewline
\end{tabular}
\end{table}

\begin{enumerate}[label=\alph*.]
    \item Show that if at least n/2 chips are bad, the professor cannot necessarily determine
          which chips are good using any strategy based on this kind of pairwise test.
          Assume that the bad chips can conspire to fool the professor

    \textbf{Answer:} 

    We will solve with mathematical induction. 
   
    \begin{itemize}
        \item \textbf{Base Case}
        
        Suppose there are only two chips, at least half
        of which are defective, i.e. one or both are defective.

        There are two possibilities.
        \begin{enumerate}
            \item Suppose both chips are defective. In that case, either both chips say that the
            other is good, or at least one says the other is bad. If both say that they are good,
            then the professor can conclude only that \textit{both are good or both are bad.}
            If at least one chip says the other is bad, then the professor can conclude only that 
            \textit{at least one is bad}.

            \item Suppose only one chips is defective. 
            In that case, the professor can conclude only that
            \textit{at least one is bad}.
        \end{enumerate}

        In either case, the professor will conclude from this test either that
        \textit{both are good or both are bad} OR that \textit{at least one is bad}.
        In other words, for $n=2$, the professor cannot determine which among his chips are good.
        This proves the base case.

        \item \textbf{Recursive Case} 
        Now suppose the professor is already uncertain about $n$ chips, at least half of which are
        defective where $n \geq 2$. If another chip was added such that $\lceil \frac{n+1}{2} \rceil$ chips are defective,
        can the professor determine which among the $n+1$ chips are good?

        The professor already cannot determine from the $n$ chips which are defective. Thus, if the professor
        cannot determine the defectiveness of the new chip, then they cannot determine the
        defectiveness of all $n+1$ chips.

        The newly added chip is either defective or it isn't:
        \begin{enumerate}
            \item
            Suppose the new chip is not defective.
            If the chip is not defective, the professor can attempt to pair this chip with one of their
            chips from the $n$ chips of indeterminate defectiveness:
            \begin{enumerate}
                \item Suppose the second chip chosen from $n$ is also not defective. Then, the professor can
                only conclude that 
                \textit{both are good or both are bad}.
                \item Suppose instead the second chip chosen from $n$ is defective. Then, the professor can
                only conclude that either \textit{both are good or both are bad} or \textit{at least one is bad}.
            \end{enumerate}

            Therefore, if the new chip is not defective, the professor once again will not be able to determine
            whether this is so.

            \item Now suppose the new chip is defective. In that case, regardless of what it is paired with,
            the professor will either conclude that 
            \textit{both are good or both are bad} or \textit{at least one is bad}.
            In other words, in this case as well, the professor cannot determine whether the new chip
            is defective.
        \end{enumerate}

        Therefore, professor will be unable to determine the defectiveness of the new chip.
        Since the professor cannot determine the defectiveness of the new chip, we can conclude
        the recursive case: that if the professor is unsure about $n$ chips where 
        $\lceil \frac{n}{2} \rceil$ chips are defective, 
        then the professor is also unsure about $n+1$ chips 
        where $\lceil \frac{n+1}{2} \rceil$ are defective.
    
    \end{itemize}

    Since the professor cannot determine the defectiveness in the base case, i.e. with two chips,
    and the professor cannot determine the defectiveness in the recursive case, then it is hopeless.
    There is no $n$ for which the professor will be able to determine the defectiveness of his chips.

    \item Consider the problem of finding a single good chip from among $n$ chips,
          assuming that more than n/2 of the chips are good. Show that $\lfloor n/2 \rfloor$
          pairwise tests are sufficient to reduce the problem to one of nearly half the size.

    \textbf{Answer:}

        We need a method to wittle down the size of
        $n$ while retaining the key property that more than half of $n$ are good chips.

        What we do is that we arbitrarily match each chip in $n$ with another chip in $n$. If there
        is an odd number of chips, then we set aside one chip from this pairing. Since we stipulate
        that more than half are good chips, if we do this one-to-one matching and we have odd chips,
        it must be the case if we leave out the one-off chip that the number of chips in
        the pairings are at least half good.

        The goal is to eliminate chips from consideration.
        When we pair these chips and pairwise test them, there are two possibilities
        and two different rules for removing chips from consideration:
        \begin{itemize}
            \item At least one chip reports that the other is bad.
            This means that \textit{at least one is bad}. If at least one
            chip is bad, then if we ignore these two chips in subsequent comparisons, i.e.
            \textbf{remove both chips from consideration completely}, then the rest of the chips must
            also still contain at least half good chips.

            \item Both chips agree that the other is good. In this case, there are two possibilities.
            Either both chips are good, or both are bad. Either way, we know that these chips share
            the same state, so we lose no information by keeping both chips in consideration.
            So, \textbf{we discard one chip from this pairwise test}.
        \end{itemize} 

        After doing this for our initial pairs, we will have some subset of our chips left, i.e.
        those that agreed with each other that the other is good. Moreover, this subset is
        guaranteed to have at least half good chips. We recursively apply this rule until
        we have one chip or no chips remaining.
        
        In the case of this algorithm for the
        odd-numbered $n$, there are two possibilities. The first is that we end up by chance pairing up
        every remaining good chip with a bad chip, thus causing all the pairs to be discarded, leaving
        just the odd-one out. In that case, the odd-one out must be good, since we stipulated 
        that more than half the chips are good. The second possibilitiy is
        that by the end of the recursion, we have one remaining chip in our pairs,
        in which case that chip must be good, since once again we stipulated that
        more than half the chips are good. Either way, we can know
        with certainty at least one good chip.

        The reason that $\lfloor n/2 \rfloor$ is sufficient to reduce the problem to one of 
        nearly half the size is that during the recursion we eliminate at least one 
        chip from consideration for each pairwise test, so each pass over the data in the worst case
        halves the number of possibilities.
        
    \item Show that the good chips can be identified with $\Theta(n)$ pairwise tests, assuming that 
          more than n/2 of the chips are good. Give and solve the recurrence that describes the
          number of tests.

    \textbf{Answer:} 
    
    Once we find a single good chip, it can be used with $n-1$ more pairwise tests
    to identify all the good and bad chips. Thus, we need only apply the solution discovered
    in part (b), then check that good chip against every other chip.

    To see that runtime is $\Theta(n)$ pairwise tests, consider that during the recursion,
    each pass over the data eliminates at least half of the possibilities and that
    the number of steps to perform during the recursion is $\lfloor n / 2 \rfloor$. Thus,
    we can express the recurrence relation as such:
    \begin{align*}
        T(N) &\leq T(\lceil N/2 \rceil) + \lfloor n / 2 \rfloor \\
        T(1) &= \Theta(1)
    \end{align*}

    By the master theorem, case 2, $T(n)$ must be $O(n)$. Once we have identified the good chip,
    we need only perform n - 1 more tests, so the total time is $O(n) + n -1$, which is $\Theta(n)$.

\end{enumerate}

\subsection*{Problem 4-6}

\textbf{Question} 
An $ m \times n$ array $A$ of real numbers is a \textbf{Monge array} if for all i, j, k, and l such
that $1 \leq i < k \leq m$ and $1 \leq j < l \leq n$, we have
\begin{align*}
    A[i, j] + A[k, l] \leq A[i, l] + A[k, j]
\end{align*}

In other words, whenever we pick two rows and two columns of a Monge array and
consider the four elements at the intersections of the rows and the columns, the sum
of the upper-left and lower-right elements is less than or equal to the sum of the
lower-left and upper-right elements. For example, the following array is Monge:

\begin{table}[H]
\centering
\captionsetup{labelfont=it,justification=centering} 		%%% Puts label font in italics, must be done for all figures %%%
\begin{tabular}{l l l l l} % <-- Alignments: 1st column left, 2nd middle and 3rd right, with vertical lines in between
    10 & 17 & 13 & 28 & 23  \tabularnewline
    17 & 22 & 16 & 29 & 23  \tabularnewline
    24 & 28 & 22 & 34 & 24  \tabularnewline
    11 & 13 & 6 & 17 & 7    \tabularnewline
    45 & 44 & 32 & 37 & 23  \tabularnewline
    36 & 33 & 19 & 21 & 6   \tabularnewline
    75 & 66 & 51 & 53 & 34  \tabularnewline
\end{tabular}
\end{table}

\begin{enumerate}[label=\alph*.]
    \item Prove that an array is Monge if and only if for all $i~=~1,~2,~...,~m~-~1$
          and $j~=~1,~2,~...,~n~-~1$, we have

    \begin{align*}
        A[i, j] + A[i+1, j+1] \leq A[i, j+1] + A[i+1, j]
    \end{align*}

    (\textit{Hint}: For the ``if'' part, use induction separately on rows and columns.)

    \textbf{Answer:}

    First, we prove that if for all $i~=~1,~2,~...,~m~-~1$
    and $j~=~1,~2,~...,~n~-~1$, then array $A$ (shown above) is Monge.

    By definition, $i$ must be positive. Therefore, $i+1$
    must be greater than $i$. Let $k = i+1$. Likewise, since $j$ is positive, $j+1$
    must be greater than $j$. Let $l = j+1$. By substituting into the above expression,
    we get the definition of a Monge.

    Now, we prove converse that if an array $A$ is Monge, then for all $i~=~1,~2,~...,~m~-~1$
    and $j~=~1,~2,~...,~n~-~1$, then array shown above is true.

    If we substitute
    $i+1$ for $k$ and $j+1$ for $l$, then we get the definition above, which shows that
    if $A$ is Monge, then the above expression holds true. We have therefore shown
    both sides of the biconditional expression, which proves the theorem.

    \item The following array is not Monge. Change one element in order to make it 
          Monge (\textit{Hint:} Use part (a).)

        \begin{table}[H]
        \centering
        \captionsetup{labelfont=it,justification=centering} 		%%% Puts label font in italics, must be done for all figures %%%
        \begin{tabular}{l l l l} % <-- Alignments: 1st column left, 2nd middle and 3rd right, with vertical lines in between
            37 & 23 & 22 & 32   \tabularnewline
            21 & 6 & 7 & 10     \tabularnewline
            53 & 34 & 30 & 31   \tabularnewline
            32 & 13 & 9 & 6     \tabularnewline
            43 & 21 & 15 & 8    \tabularnewline
        \end{tabular}
        \end{table}

    \textbf{Answer:} 

        \begin{table}[H]
        \centering
        \captionsetup{labelfont=it,justification=centering} 		%%% Puts label font in italics, must be done for all figures %%%
        \begin{tabular}{l l l l} % <-- Alignments: 1st column left, 2nd middle and 3rd right, with vertical lines in between
            37 & 23 & \textbf{24} & 32   \tabularnewline
            21 & 6 & 7 & 10     \tabularnewline
            53 & 34 & 30 & 31   \tabularnewline
            32 & 13 & 9 & 6     \tabularnewline
            43 & 21 & 15 & 8    \tabularnewline
        \end{tabular}
        \end{table}

    \item Let $f(i)$ be the index of the column containing the leftmost minimum element of row $i$.
          Prove that $f(1) \leq f(2) \leq \dots \leq f(m)$ for any $m \times n$ Monge array.

    \textbf{Answer:} 

    From the theorem proved in (a), we can say that:
    \begin{align*}
        A[i, f(i)] + A[i+1, f(i)+1] \leq A[i+1, f(i)] + A[i, f(i)+1]
    \end{align*}

    In other words, whatever the index column of the smallest element in row $i$, it will
    be the case that the Monge property holds for the next index.

    Now if it's true that $f(i)$ is the index of the column containing the leftmost minimum element of row $i$,
    then the following is also true:
    \begin{align*}
        A[i, f(i)] \leq A[i, f(i)+1]
    \end{align*}
    
    In other words, the value immediately to the right of $A[i, f(i)]$
    is obviously larger or equal if $A[i, f(i)]$ is the smallest minimum value in row $i$.
    
    We can then subtract $A[i, f(i)]$ from the left-hand side of the above equation and subtract
    $A[i, f(i)+1]$ from the right-hand side, since this will maintain the truth of the inequality:

    \begin{align*}
        A[i, f(i)] + A[i+1, f(i)+1] &\leq A[i+1, f(i)] + A[i, f(i)+1] \\
        A[i, f(i)] + A[i+1, f(i)+1] - A[i, f(i)] &\leq A[i, f(i)+1] + A[i+1, f(i)] - A[i, f(i)+1] \\
        A[i+1, f(i)+1] &\leq A[i+1, f(i)]
    \end{align*}

    This statement, $A[i+1, f(i)+1] \leq A[i+1, f(i)]$, proves our theorem, since it says 
    in the next row, the value to the right of $f(i)$ must be less than or equal to the value
    at $f(i)$, which is to say that the index of the minimum smallest value in the next row
    must be greater than or equal to the index in the current row. In other words, for any
    $f(i)$, $f(i) \leq f(i+1)$.

    \item Here is a description of a divide-and-conquer algorithm that computes the leftmost minimum
          element in each row of an $m \times n$ Monge array $A$:
          \begin{itemize}
            \item Construct a submatrix $A'$ of $A$ consisting of the even-numbered rows of $A$.
                  Recursively determine the leftmost minimum for each row of $A'$.
                  Then compute the leftmost minimum in the odd-numbered rows of $A$.
          \end{itemize}
          Explain how to compute the leftmost minimum in the odd-numbered rows of A
          (given that the leftmost minimum of the even-numbered rows is known) in $O(m+n)$.

    \textbf{Answer:} 

          Once the indices of the minimum values in the even-numbered rows
          are known, we do not have to check all columns for the odd-numbered rows.
          We know this because of the theorem we proved in part (c). In fact,
          the only columns we need to check for row $i$ are the columns
          between $f(i-1)$ and $f(i+1)$, since the theorem in part (c) proved
          that $f(i-1) \leq f(i) \leq f(i+1)$ where $i>1$ and $f(i)$ is the
          function of the $i$th row that returns the index of the minimum leftmost value in
          row $i$.

          Let even\_minimums contain the leftmost index columns, [f(0), f(2), f(4)...f(m)], for 
          an $m \times n$ matrix. A simple algorithm to find the
          minimums for the odd rows, [f(1), f(3), f(5)...] is the following:

            \begin{algorithm}[H]
            \caption{get\_odd\_mins(A, even\_mins)}
            \begin{algorithmic}[1]
            \STATE odds $\leftarrow$ odd rows from A
            \STATE odd\_mins $\leftarrow$ []
            \FOR{i, row in enumerate(odds)}
                \STATE colstart $\leftarrow$ even\_mins[i]
                \STATE colend $\leftarrow$ even\_mins[i+1]
                \STATE mincol $\leftarrow$ colstart
                \STATE min $\leftarrow$ row[mincol]
                \FOR{i in range(colstart, colend)}
                    \IF{row[i] $<$ min}
                        \STATE mincol $\leftarrow$ i
                        \STATE min $\leftarrow$ row[i]
                    \ENDIF
                \ENDFOR
                \STATE odd\_mins $\leftarrow$ odd\_mins + [mincol]
            \ENDFOR
            \STATE return odd\_mins
            \end{algorithmic}
            \end{algorithm}

        In other words, we only check the columns between $f(i-1)$ and $f(i+1)$ since the
        smallest value must be between those columns.

        The for loop iterates over $\lfloor m/2 \rfloor$ elements,
        doing constant time operations until we hit the inner for loop.
        This inner for-loop will iterate over a certain subset of
        columns. Specifically, each for-loop will iterate over
        a unique subset of columns, since the column position
        of the minimum value in each row is ``sandwhiched'' between
        two columns. In other words, if we sum the number of
        loops over columns across the entire outer for loop,
        we would find a total of $n$ steps. Thus,
        the outer for loop can be thought of as adding 
        $\lfloor m/2 \rfloor$ steps plus $n$ steps, which
        is to say, $O(m+n)$.

    \item Write the recurrence describing the running time of the algorithm described in part (d).
          Show that its solution is $O(m + n\log m)$

        \textbf{Answer:} 

          The algorithm described in part (d) involves finding the minimums of the evens,
          then finding the minimums of the odds. The running time to find the minimum
          of the odds is $O(m+n$), thus we need only to find the running time
          to find the minimum of the evens. The algorithm involves
          a simple divide-and-conquer approach, which halves the number 
          of elements in the matrix. Thus, our recurrence is the following:

            % \begin{algorithm}[H]
            % \caption{get\_even\_mins(A, even\_mins)}
            % \begin{algorithmic}[1]
            % \STATE evens $\leftarrow$ even rows from A
            % \STATE return get\_mins(evens)
            % \end{algorithmic}
            % \end{algorithm}

            % \begin{algorithm}[H]
            % \caption{get\_mins(A, colstart=0)}
            % \begin{algorithmic}[1]
            % \STATE numrows $\leftarrow$ len(A)
            % \STATE numcols $\leftarrow$ len(A[0])
            % \STATE \# Base Case
            % \IF{numrows == 1}
            %     \STATE mincol $\leftarrow$ colstart
            %     \STATE min $\leftarrow$ A[0][colstart]
            %     \FOR{i in range(colstart, numcols)}
            %         \IF{A[0][i] $<$ min}
            %             \STATE mincol $\leftarrow$ i
            %             \STATE min $\leftarrow$ A[0][mincol]
            %         \ENDIF
            %     \ENDFOR
            %     \STATE return [mincol]
            % \ENDIF
            % \STATE \# Recursive Case
            % \STATE midrow $\leftarrow$ int(numrows / 2)
            % \STATE abovemins $\leftarrow$ get\_mins(A[:midrow], colstart=colstart)
            % \STATE belowmins $\leftarrow$ get\_mins(A[midrow:], colstart=abovemins[-1])
            % \STATE return abovemins + belowmins
            % \end{algorithmic}
            % \end{algorithm}

            % Analyzing this algorithm into a recurrence relation is tricky.
            % The base case is dependent on the index of the starting column
            % to check. For the first recursive call, this value will be 0,
            % and in subsequent calls, it will start to increase, thus reducing
            % the work needed in the base case. In fact, each time the base case
            % is reached, there will be unique column ranges to check, so
            % once again we can think of this loop over the columns in the base
            % case as collectively taking $\Theta(n)$ steps. Since we
            % also have to loop over all $m$ rows to retrieve the odd and
            % even-numbered rows, we should also add that and make the
            % time in addition to the recursive calls, $\Theta(m + n)$.

            % The recurrence for this algorithm is as follows:

            \begin{align*}
                T(m) &= T(m/2) + \Theta(m + n) \\
                T(m) &= T(m/2) + mc_1 + nc_2
            \end{align*}

            First, we unroll with different inputs:
        
            \begin{align*}
                T(m)   &= T(m/2) + mc_1 + nc_2 \\
                T(m/2) &= T(m/4) + \frac{m}{2}c_1 + nc_2 \\
                T(m/4) &= T(m/8) + \frac{m}{4}c_1 + nc_2 \\
                        &...
            \end{align*}

        We substitute recursively to unroll $T(n)$:

        \begin{align*}
          T(m) &= mc_1 + nc_2 + T(m/2)\\
               &= \frac{m}{2}c_1 + mc_1 + nc_2 + nc_2 + T(m/2^2)\\
               &= \frac{m}{4}c_1 + \frac{m}{2}c_1 + mc_1 + nc_2 + nc_2 + nc_2 + T(m/2^3)\\
               &= mc_1 \sum_{i=0}^{k-1} \frac{1}{2^k} + k nc_2 + T(m/2^k) \\
               &= mc_1 (2 - \frac{2}{2^k}) + k nc_2 + T(m/2^k) \\
               &= mc_1 (2 - \frac{2}{m}) + \log_2 m nc_2 + T(1) \\
               &\leq mc_1 + nc_2 \log_2 m + \Theta(1) \\
        \end{align*}

        Thus, the running time of $T(m)$ is $O(m + n \log m)$.
            
\end{enumerate}


\subsection*{Leetcode 1}

    \textbf{Answer:} See leetcode1.txt

\subsection*{Leetcode 2}

    \textbf{Answer:} See leetcode2.txt

\end{document}