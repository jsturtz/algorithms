\documentclass[]{book}

%These tell TeX which packages to use.
\usepackage{array,epsfig}
\usepackage{amsmath}
\usepackage{amsfonts}
\usepackage{amssymb}
\usepackage{amsxtra}
\usepackage{amsthm}
\usepackage{mathrsfs}
\usepackage{color}

% JORDAN ADDED THESE
\usepackage{enumerate}
\usepackage[shortlabels]{enumitem}
% -----------------------------------------

% PACKAGES I ADDED START HERE
% \usepackage{minted}
\usepackage{listings}
\usepackage{hyperref}
\usepackage{xcolor}
\hypersetup{
    colorlinks=true,
    linkcolor=black,
    citecolor=black,
    urlcolor=blue,
    pdftitle={ScaffoldFiller},
    pdfpagemode=FullScreen,
}

\definecolor{codegreen}{rgb}{0,0.6,0}
\definecolor{codegray}{rgb}{0.5,0.5,0.5}
\definecolor{codepurple}{rgb}{0.58,0,0.82}
\definecolor{backcolour}{rgb}{0.95,0.95,0.92}

\lstdefinestyle{mystyle}{
    backgroundcolor=\color{backcolour},   
    commentstyle=\color{codegreen},
    keywordstyle=\color{magenta},
    numberstyle=\tiny\color{codegray},
    stringstyle=\color{codepurple},
    basicstyle=\ttfamily\footnotesize,
    breakatwhitespace=false,         
    breaklines=true,                 
    captionpos=b,                    
    keepspaces=true,                 
    numbers=left,                    
    numbersep=5pt,                  
    showspaces=false,                
    showstringspaces=false,
    showtabs=false,                  
    tabsize=2
}

\lstset{style=mystyle}
\renewcommand{\lstlistingname}{Algorithm}


%Here I define some theorem styles and shortcut commands for symbols I use often
\theoremstyle{definition}
\newtheorem{defn}{Definition}
\newtheorem{thm}{Theorem}
\newtheorem{cor}{Corollary}
\newtheorem*{rmk}{Remark}
\newtheorem{lem}{Lemma}
\newtheorem*{joke}{Joke}
\newtheorem{ex}{Example}
\newtheorem*{soln}{Solution}
\newtheorem{prop}{Proposition}

\newcommand{\lra}{\longrightarrow}
\newcommand{\ra}{\rightarrow}
\newcommand{\surj}{\twoheadrightarrow}
\newcommand{\graph}{\mathrm{graph}}
\newcommand{\bb}[1]{\mathbb{#1}}
\newcommand{\Z}{\bb{Z}}
\newcommand{\Q}{\bb{Q}}
\newcommand{\R}{\bb{R}}
\newcommand{\C}{\bb{C}}
\newcommand{\N}{\bb{N}}
\newcommand{\M}{\mathbf{M}}
\newcommand{\m}{\mathbf{m}}
\newcommand{\MM}{\mathscr{M}}
\newcommand{\HH}{\mathscr{H}}
\newcommand{\Om}{\Omega}
\newcommand{\Ho}{\in\HH(\Om)}
\newcommand{\bd}{\partial}
\newcommand{\del}{\partial}
\newcommand{\bardel}{\overline\partial}
\newcommand{\textdf}[1]{\textbf{\textsf{#1}}\index{#1}}
\newcommand{\img}{\mathrm{img}}
\newcommand{\ip}[2]{\left\langle{#1},{#2}\right\rangle}
\newcommand{\inter}[1]{\mathrm{int}{#1}}
\newcommand{\exter}[1]{\mathrm{ext}{#1}}
\newcommand{\cl}[1]{\mathrm{cl}{#1}}
\newcommand{\ds}{\displaystyle}
\newcommand{\vol}{\mathrm{vol}}
\newcommand{\cnt}{\mathrm{ct}}
\newcommand{\osc}{\mathrm{osc}}
\newcommand{\LL}{\mathbf{L}}
\newcommand{\UU}{\mathbf{U}}
\newcommand{\support}{\mathrm{support}}
\newcommand{\AND}{\;\wedge\;}
\newcommand{\OR}{\;\vee\;}
\newcommand{\Oset}{\varnothing}
\newcommand{\st}{\ni}
\newcommand{\wh}{\widehat}

%Pagination stuff.
\setlength{\topmargin}{-.3 in}
\setlength{\oddsidemargin}{0in}
\setlength{\evensidemargin}{0in}
\setlength{\textheight}{9.in}
\setlength{\textwidth}{6.5in}
\pagestyle{empty}



\begin{document}


\begin{center}
{\Large COMP 775 - Advanced Design and Analysis of Algorithms \hspace{0.5cm} Assignment 3}\\
\textbf{Jordan Sturtz}\\ %You should put your name here
2-19-2023 %You should write the date here.
\end{center}

\vspace{0.2 cm}

\subsection*{Problem 15-2}

\textbf{Question} 
A palindrome is a nonempty string over some alphabet that reads the same forward and backward. Examples of palindromes are all strings of length 1, civic, racecar, and aibohphobia (fear of palindromes).
Give an efficient algorithm to find the longest palindrome that is a subsequence of a given input string. For example, given the input \textit{character}, your algorithm should return \textit{carac}. What is the running time of your algorithm?

\textbf{Answer:} 

\begin{lstlisting}[language=Python, caption=Longest Palindrome Subsequence without DP, label=snip:withoutdp]
def longest_palindrome_subsequence(s):

    # Base case: string has one character in it
    if len(s) == 1:
        return s

    # Recursive case
    if s[0] == s[-1]:
        middle = longest_palindrome_subsequence(s[1:-1])
        return s[0] + middle + s[-1]
    else:
        left = longest_palindrome_subsequence(s[1:])
        right = longest_palindrome_subsequence(s[:-1])

        val = left if len(left) > len(right) else right
        return val\end{lstlisting}

\begin{lstlisting}[language=Python, caption=Longest Palindrome Subsequence with Memoization, label=snip:withdp]
def longest_palindrome_subsequence_dp(s):

    def recursive_helper(s, memo):

        # Don't recompute something we've memoized!
        if s in memo:
            return memo[s]

        # Base case: string has one character in it
        if len(s) == 1:
            return s

        # Recursive case
        if s[0] == s[-1]:
            middle = recursive_helper(s[1:-1], memo)
            val = s[0] + middle + s[-1]
            memo[s] = val
            return val
        else:
            left = recursive_helper(s[1:], memo)
            right = recursive_helper(s[:-1], memo)

            val = left if len(left) > len(right) else right
            memo[s] = val
            return val

    return recursive_helper(s, {})\end{lstlisting}

    \clearpage

\begin{lstlisting}[language=Python, caption=Longest Palindrome Subsequence Bottom Up DP, label=snip:bottomup]
def longest_palindrome_subsequence_bottom_up(s):

    # the longest palindrome for every c in s is c itself
    memo = {c: c for c in s}
    memo[""] = ""

    # we slide over s with window to compute subproblems in bottom-up fashion
    for window in range(2, len(s) + 1):
        for i in range(len(s) - window + 1):
            subproblem = s[i:i+window]

            if subproblem[0] == subproblem[-1]:
                palindrome = subproblem[0] + memo[subproblem[1:-1]] + subproblem[-1]
            else:
                left = memo[subproblem[1:]]
                right = memo[subproblem[:-1]]
                palindrome = left if len(left) > len(right) else right

            memo[subproblem] = palindrome

    return memo[s]\end{lstlisting}

    From the bottom-up implementation in Algorithm~\ref{snip:bottomup},
    we can see that the algorithmic complexity is $O(n^2)$ where
    $n$ is the length of $s$, since the algorithm involves a nested for loop over $s$.

\subsection*{Problem 15-4}

\textbf{Question} 
    Consider the problem of neatly printing a paragraph with a monospaced font (all
    characters having the same width) on a printer. The input text is a sequence of
    $n$ words of lengths $l_1$, $l_2$, $\dots l_n$ measured in characters.

    We want to print this paragraph neatly on a number of lines that hold a 
    maximum of M characters each. 
    Our criterion of ``neatness'' is as follows. 
    If a given line contains words i through j, where $i \leq j$,
    and we leave exactly one space between words, the number of extra
    space characters at the end of the line is
    $M - j + i - \sum_{k=i}^j l_k$,
    which must be nonnegative so that the words fit on the line. 
    We wish to minimize the sum, over
    all lines except the last, of the cubes of the numbers of extra space characters at the
    ends of lines. Give a dynamic-programming algorithm to print a paragraph of n
    words neatly on a printer. Analyze the running time and space requirements of
    your algorithm.

\textbf{Answer}

    The optimal substructure in this problem is simple: if the optimal
    amount of words that can fit on the first $k$ lines is $j$ where $j < n$,
    then we need only neatly print words from $j+1$ to $n$. That would
    produce the optimal solution.

    To measure the space complexity, we assume the goal is to store the
    minimum information necessary to print via indices. Thus,
    we will store a dictionary, $R$, mapping row indices to the index
    of the last word that should be printed on that row. To find the
    optimal solution for the $k$th line, we must find the solution
    for the $k-1th$ line. So we build up the problem in a bottom-up
    fashion from the first line.



FIXME: Give running time and space requirements of algorithm



\end{document}